
% Default to the notebook output style

    


% Inherit from the specified cell style.




    
\documentclass[11pt]{article}

    
    
    \usepackage[T1]{fontenc}
    % Nicer default font (+ math font) than Computer Modern for most use cases
    \usepackage{mathpazo}

    % Basic figure setup, for now with no caption control since it's done
    % automatically by Pandoc (which extracts ![](path) syntax from Markdown).
    \usepackage{graphicx}
    % We will generate all images so they have a width \maxwidth. This means
    % that they will get their normal width if they fit onto the page, but
    % are scaled down if they would overflow the margins.
    \makeatletter
    \def\maxwidth{\ifdim\Gin@nat@width>\linewidth\linewidth
    \else\Gin@nat@width\fi}
    \makeatother
    \let\Oldincludegraphics\includegraphics
    % Set max figure width to be 80% of text width, for now hardcoded.
    \renewcommand{\includegraphics}[1]{\Oldincludegraphics[width=.8\maxwidth]{#1}}
    % Ensure that by default, figures have no caption (until we provide a
    % proper Figure object with a Caption API and a way to capture that
    % in the conversion process - todo).
    \usepackage{caption}
    \DeclareCaptionLabelFormat{nolabel}{}
    \captionsetup{labelformat=nolabel}

    \usepackage{adjustbox} % Used to constrain images to a maximum size 
    \usepackage{xcolor} % Allow colors to be defined
    \usepackage{enumerate} % Needed for markdown enumerations to work
    \usepackage{geometry} % Used to adjust the document margins
    \usepackage{amsmath} % Equations
    \usepackage{amssymb} % Equations
    \usepackage{textcomp} % defines textquotesingle
    % Hack from http://tex.stackexchange.com/a/47451/13684:
    \AtBeginDocument{%
        \def\PYZsq{\textquotesingle}% Upright quotes in Pygmentized code
    }
    \usepackage{upquote} % Upright quotes for verbatim code
    \usepackage{eurosym} % defines \euro
    \usepackage[mathletters]{ucs} % Extended unicode (utf-8) support
    \usepackage[utf8x]{inputenc} % Allow utf-8 characters in the tex document
    \usepackage{fancyvrb} % verbatim replacement that allows latex
    \usepackage{grffile} % extends the file name processing of package graphics 
                         % to support a larger range 
    % The hyperref package gives us a pdf with properly built
    % internal navigation ('pdf bookmarks' for the table of contents,
    % internal cross-reference links, web links for URLs, etc.)
    \usepackage{hyperref}
    \usepackage{longtable} % longtable support required by pandoc >1.10
    \usepackage{booktabs}  % table support for pandoc > 1.12.2
    \usepackage[inline]{enumitem} % IRkernel/repr support (it uses the enumerate* environment)
    \usepackage[normalem]{ulem} % ulem is needed to support strikethroughs (\sout)
                                % normalem makes italics be italics, not underlines
    

    
    
    % Colors for the hyperref package
    \definecolor{urlcolor}{rgb}{0,.145,.698}
    \definecolor{linkcolor}{rgb}{.71,0.21,0.01}
    \definecolor{citecolor}{rgb}{.12,.54,.11}

    % ANSI colors
    \definecolor{ansi-black}{HTML}{3E424D}
    \definecolor{ansi-black-intense}{HTML}{282C36}
    \definecolor{ansi-red}{HTML}{E75C58}
    \definecolor{ansi-red-intense}{HTML}{B22B31}
    \definecolor{ansi-green}{HTML}{00A250}
    \definecolor{ansi-green-intense}{HTML}{007427}
    \definecolor{ansi-yellow}{HTML}{DDB62B}
    \definecolor{ansi-yellow-intense}{HTML}{B27D12}
    \definecolor{ansi-blue}{HTML}{208FFB}
    \definecolor{ansi-blue-intense}{HTML}{0065CA}
    \definecolor{ansi-magenta}{HTML}{D160C4}
    \definecolor{ansi-magenta-intense}{HTML}{A03196}
    \definecolor{ansi-cyan}{HTML}{60C6C8}
    \definecolor{ansi-cyan-intense}{HTML}{258F8F}
    \definecolor{ansi-white}{HTML}{C5C1B4}
    \definecolor{ansi-white-intense}{HTML}{A1A6B2}

    % commands and environments needed by pandoc snippets
    % extracted from the output of `pandoc -s`
    \providecommand{\tightlist}{%
      \setlength{\itemsep}{0pt}\setlength{\parskip}{0pt}}
    \DefineVerbatimEnvironment{Highlighting}{Verbatim}{commandchars=\\\{\}}
    % Add ',fontsize=\small' for more characters per line
    \newenvironment{Shaded}{}{}
    \newcommand{\KeywordTok}[1]{\textcolor[rgb]{0.00,0.44,0.13}{\textbf{{#1}}}}
    \newcommand{\DataTypeTok}[1]{\textcolor[rgb]{0.56,0.13,0.00}{{#1}}}
    \newcommand{\DecValTok}[1]{\textcolor[rgb]{0.25,0.63,0.44}{{#1}}}
    \newcommand{\BaseNTok}[1]{\textcolor[rgb]{0.25,0.63,0.44}{{#1}}}
    \newcommand{\FloatTok}[1]{\textcolor[rgb]{0.25,0.63,0.44}{{#1}}}
    \newcommand{\CharTok}[1]{\textcolor[rgb]{0.25,0.44,0.63}{{#1}}}
    \newcommand{\StringTok}[1]{\textcolor[rgb]{0.25,0.44,0.63}{{#1}}}
    \newcommand{\CommentTok}[1]{\textcolor[rgb]{0.38,0.63,0.69}{\textit{{#1}}}}
    \newcommand{\OtherTok}[1]{\textcolor[rgb]{0.00,0.44,0.13}{{#1}}}
    \newcommand{\AlertTok}[1]{\textcolor[rgb]{1.00,0.00,0.00}{\textbf{{#1}}}}
    \newcommand{\FunctionTok}[1]{\textcolor[rgb]{0.02,0.16,0.49}{{#1}}}
    \newcommand{\RegionMarkerTok}[1]{{#1}}
    \newcommand{\ErrorTok}[1]{\textcolor[rgb]{1.00,0.00,0.00}{\textbf{{#1}}}}
    \newcommand{\NormalTok}[1]{{#1}}
    
    % Additional commands for more recent versions of Pandoc
    \newcommand{\ConstantTok}[1]{\textcolor[rgb]{0.53,0.00,0.00}{{#1}}}
    \newcommand{\SpecialCharTok}[1]{\textcolor[rgb]{0.25,0.44,0.63}{{#1}}}
    \newcommand{\VerbatimStringTok}[1]{\textcolor[rgb]{0.25,0.44,0.63}{{#1}}}
    \newcommand{\SpecialStringTok}[1]{\textcolor[rgb]{0.73,0.40,0.53}{{#1}}}
    \newcommand{\ImportTok}[1]{{#1}}
    \newcommand{\DocumentationTok}[1]{\textcolor[rgb]{0.73,0.13,0.13}{\textit{{#1}}}}
    \newcommand{\AnnotationTok}[1]{\textcolor[rgb]{0.38,0.63,0.69}{\textbf{\textit{{#1}}}}}
    \newcommand{\CommentVarTok}[1]{\textcolor[rgb]{0.38,0.63,0.69}{\textbf{\textit{{#1}}}}}
    \newcommand{\VariableTok}[1]{\textcolor[rgb]{0.10,0.09,0.49}{{#1}}}
    \newcommand{\ControlFlowTok}[1]{\textcolor[rgb]{0.00,0.44,0.13}{\textbf{{#1}}}}
    \newcommand{\OperatorTok}[1]{\textcolor[rgb]{0.40,0.40,0.40}{{#1}}}
    \newcommand{\BuiltInTok}[1]{{#1}}
    \newcommand{\ExtensionTok}[1]{{#1}}
    \newcommand{\PreprocessorTok}[1]{\textcolor[rgb]{0.74,0.48,0.00}{{#1}}}
    \newcommand{\AttributeTok}[1]{\textcolor[rgb]{0.49,0.56,0.16}{{#1}}}
    \newcommand{\InformationTok}[1]{\textcolor[rgb]{0.38,0.63,0.69}{\textbf{\textit{{#1}}}}}
    \newcommand{\WarningTok}[1]{\textcolor[rgb]{0.38,0.63,0.69}{\textbf{\textit{{#1}}}}}
    
    
    % Define a nice break command that doesn't care if a line doesn't already
    % exist.
    \def\br{\hspace*{\fill} \\* }
    % Math Jax compatability definitions
    \def\gt{>}
    \def\lt{<}
    % Document parameters
    \title{02vectors}
    
    
    

    % Pygments definitions
    
\makeatletter
\def\PY@reset{\let\PY@it=\relax \let\PY@bf=\relax%
    \let\PY@ul=\relax \let\PY@tc=\relax%
    \let\PY@bc=\relax \let\PY@ff=\relax}
\def\PY@tok#1{\csname PY@tok@#1\endcsname}
\def\PY@toks#1+{\ifx\relax#1\empty\else%
    \PY@tok{#1}\expandafter\PY@toks\fi}
\def\PY@do#1{\PY@bc{\PY@tc{\PY@ul{%
    \PY@it{\PY@bf{\PY@ff{#1}}}}}}}
\def\PY#1#2{\PY@reset\PY@toks#1+\relax+\PY@do{#2}}

\expandafter\def\csname PY@tok@w\endcsname{\def\PY@tc##1{\textcolor[rgb]{0.73,0.73,0.73}{##1}}}
\expandafter\def\csname PY@tok@c\endcsname{\let\PY@it=\textit\def\PY@tc##1{\textcolor[rgb]{0.25,0.50,0.50}{##1}}}
\expandafter\def\csname PY@tok@cp\endcsname{\def\PY@tc##1{\textcolor[rgb]{0.74,0.48,0.00}{##1}}}
\expandafter\def\csname PY@tok@k\endcsname{\let\PY@bf=\textbf\def\PY@tc##1{\textcolor[rgb]{0.00,0.50,0.00}{##1}}}
\expandafter\def\csname PY@tok@kp\endcsname{\def\PY@tc##1{\textcolor[rgb]{0.00,0.50,0.00}{##1}}}
\expandafter\def\csname PY@tok@kt\endcsname{\def\PY@tc##1{\textcolor[rgb]{0.69,0.00,0.25}{##1}}}
\expandafter\def\csname PY@tok@o\endcsname{\def\PY@tc##1{\textcolor[rgb]{0.40,0.40,0.40}{##1}}}
\expandafter\def\csname PY@tok@ow\endcsname{\let\PY@bf=\textbf\def\PY@tc##1{\textcolor[rgb]{0.67,0.13,1.00}{##1}}}
\expandafter\def\csname PY@tok@nb\endcsname{\def\PY@tc##1{\textcolor[rgb]{0.00,0.50,0.00}{##1}}}
\expandafter\def\csname PY@tok@nf\endcsname{\def\PY@tc##1{\textcolor[rgb]{0.00,0.00,1.00}{##1}}}
\expandafter\def\csname PY@tok@nc\endcsname{\let\PY@bf=\textbf\def\PY@tc##1{\textcolor[rgb]{0.00,0.00,1.00}{##1}}}
\expandafter\def\csname PY@tok@nn\endcsname{\let\PY@bf=\textbf\def\PY@tc##1{\textcolor[rgb]{0.00,0.00,1.00}{##1}}}
\expandafter\def\csname PY@tok@ne\endcsname{\let\PY@bf=\textbf\def\PY@tc##1{\textcolor[rgb]{0.82,0.25,0.23}{##1}}}
\expandafter\def\csname PY@tok@nv\endcsname{\def\PY@tc##1{\textcolor[rgb]{0.10,0.09,0.49}{##1}}}
\expandafter\def\csname PY@tok@no\endcsname{\def\PY@tc##1{\textcolor[rgb]{0.53,0.00,0.00}{##1}}}
\expandafter\def\csname PY@tok@nl\endcsname{\def\PY@tc##1{\textcolor[rgb]{0.63,0.63,0.00}{##1}}}
\expandafter\def\csname PY@tok@ni\endcsname{\let\PY@bf=\textbf\def\PY@tc##1{\textcolor[rgb]{0.60,0.60,0.60}{##1}}}
\expandafter\def\csname PY@tok@na\endcsname{\def\PY@tc##1{\textcolor[rgb]{0.49,0.56,0.16}{##1}}}
\expandafter\def\csname PY@tok@nt\endcsname{\let\PY@bf=\textbf\def\PY@tc##1{\textcolor[rgb]{0.00,0.50,0.00}{##1}}}
\expandafter\def\csname PY@tok@nd\endcsname{\def\PY@tc##1{\textcolor[rgb]{0.67,0.13,1.00}{##1}}}
\expandafter\def\csname PY@tok@s\endcsname{\def\PY@tc##1{\textcolor[rgb]{0.73,0.13,0.13}{##1}}}
\expandafter\def\csname PY@tok@sd\endcsname{\let\PY@it=\textit\def\PY@tc##1{\textcolor[rgb]{0.73,0.13,0.13}{##1}}}
\expandafter\def\csname PY@tok@si\endcsname{\let\PY@bf=\textbf\def\PY@tc##1{\textcolor[rgb]{0.73,0.40,0.53}{##1}}}
\expandafter\def\csname PY@tok@se\endcsname{\let\PY@bf=\textbf\def\PY@tc##1{\textcolor[rgb]{0.73,0.40,0.13}{##1}}}
\expandafter\def\csname PY@tok@sr\endcsname{\def\PY@tc##1{\textcolor[rgb]{0.73,0.40,0.53}{##1}}}
\expandafter\def\csname PY@tok@ss\endcsname{\def\PY@tc##1{\textcolor[rgb]{0.10,0.09,0.49}{##1}}}
\expandafter\def\csname PY@tok@sx\endcsname{\def\PY@tc##1{\textcolor[rgb]{0.00,0.50,0.00}{##1}}}
\expandafter\def\csname PY@tok@m\endcsname{\def\PY@tc##1{\textcolor[rgb]{0.40,0.40,0.40}{##1}}}
\expandafter\def\csname PY@tok@gh\endcsname{\let\PY@bf=\textbf\def\PY@tc##1{\textcolor[rgb]{0.00,0.00,0.50}{##1}}}
\expandafter\def\csname PY@tok@gu\endcsname{\let\PY@bf=\textbf\def\PY@tc##1{\textcolor[rgb]{0.50,0.00,0.50}{##1}}}
\expandafter\def\csname PY@tok@gd\endcsname{\def\PY@tc##1{\textcolor[rgb]{0.63,0.00,0.00}{##1}}}
\expandafter\def\csname PY@tok@gi\endcsname{\def\PY@tc##1{\textcolor[rgb]{0.00,0.63,0.00}{##1}}}
\expandafter\def\csname PY@tok@gr\endcsname{\def\PY@tc##1{\textcolor[rgb]{1.00,0.00,0.00}{##1}}}
\expandafter\def\csname PY@tok@ge\endcsname{\let\PY@it=\textit}
\expandafter\def\csname PY@tok@gs\endcsname{\let\PY@bf=\textbf}
\expandafter\def\csname PY@tok@gp\endcsname{\let\PY@bf=\textbf\def\PY@tc##1{\textcolor[rgb]{0.00,0.00,0.50}{##1}}}
\expandafter\def\csname PY@tok@go\endcsname{\def\PY@tc##1{\textcolor[rgb]{0.53,0.53,0.53}{##1}}}
\expandafter\def\csname PY@tok@gt\endcsname{\def\PY@tc##1{\textcolor[rgb]{0.00,0.27,0.87}{##1}}}
\expandafter\def\csname PY@tok@err\endcsname{\def\PY@bc##1{\setlength{\fboxsep}{0pt}\fcolorbox[rgb]{1.00,0.00,0.00}{1,1,1}{\strut ##1}}}
\expandafter\def\csname PY@tok@kc\endcsname{\let\PY@bf=\textbf\def\PY@tc##1{\textcolor[rgb]{0.00,0.50,0.00}{##1}}}
\expandafter\def\csname PY@tok@kd\endcsname{\let\PY@bf=\textbf\def\PY@tc##1{\textcolor[rgb]{0.00,0.50,0.00}{##1}}}
\expandafter\def\csname PY@tok@kn\endcsname{\let\PY@bf=\textbf\def\PY@tc##1{\textcolor[rgb]{0.00,0.50,0.00}{##1}}}
\expandafter\def\csname PY@tok@kr\endcsname{\let\PY@bf=\textbf\def\PY@tc##1{\textcolor[rgb]{0.00,0.50,0.00}{##1}}}
\expandafter\def\csname PY@tok@bp\endcsname{\def\PY@tc##1{\textcolor[rgb]{0.00,0.50,0.00}{##1}}}
\expandafter\def\csname PY@tok@fm\endcsname{\def\PY@tc##1{\textcolor[rgb]{0.00,0.00,1.00}{##1}}}
\expandafter\def\csname PY@tok@vc\endcsname{\def\PY@tc##1{\textcolor[rgb]{0.10,0.09,0.49}{##1}}}
\expandafter\def\csname PY@tok@vg\endcsname{\def\PY@tc##1{\textcolor[rgb]{0.10,0.09,0.49}{##1}}}
\expandafter\def\csname PY@tok@vi\endcsname{\def\PY@tc##1{\textcolor[rgb]{0.10,0.09,0.49}{##1}}}
\expandafter\def\csname PY@tok@vm\endcsname{\def\PY@tc##1{\textcolor[rgb]{0.10,0.09,0.49}{##1}}}
\expandafter\def\csname PY@tok@sa\endcsname{\def\PY@tc##1{\textcolor[rgb]{0.73,0.13,0.13}{##1}}}
\expandafter\def\csname PY@tok@sb\endcsname{\def\PY@tc##1{\textcolor[rgb]{0.73,0.13,0.13}{##1}}}
\expandafter\def\csname PY@tok@sc\endcsname{\def\PY@tc##1{\textcolor[rgb]{0.73,0.13,0.13}{##1}}}
\expandafter\def\csname PY@tok@dl\endcsname{\def\PY@tc##1{\textcolor[rgb]{0.73,0.13,0.13}{##1}}}
\expandafter\def\csname PY@tok@s2\endcsname{\def\PY@tc##1{\textcolor[rgb]{0.73,0.13,0.13}{##1}}}
\expandafter\def\csname PY@tok@sh\endcsname{\def\PY@tc##1{\textcolor[rgb]{0.73,0.13,0.13}{##1}}}
\expandafter\def\csname PY@tok@s1\endcsname{\def\PY@tc##1{\textcolor[rgb]{0.73,0.13,0.13}{##1}}}
\expandafter\def\csname PY@tok@mb\endcsname{\def\PY@tc##1{\textcolor[rgb]{0.40,0.40,0.40}{##1}}}
\expandafter\def\csname PY@tok@mf\endcsname{\def\PY@tc##1{\textcolor[rgb]{0.40,0.40,0.40}{##1}}}
\expandafter\def\csname PY@tok@mh\endcsname{\def\PY@tc##1{\textcolor[rgb]{0.40,0.40,0.40}{##1}}}
\expandafter\def\csname PY@tok@mi\endcsname{\def\PY@tc##1{\textcolor[rgb]{0.40,0.40,0.40}{##1}}}
\expandafter\def\csname PY@tok@il\endcsname{\def\PY@tc##1{\textcolor[rgb]{0.40,0.40,0.40}{##1}}}
\expandafter\def\csname PY@tok@mo\endcsname{\def\PY@tc##1{\textcolor[rgb]{0.40,0.40,0.40}{##1}}}
\expandafter\def\csname PY@tok@ch\endcsname{\let\PY@it=\textit\def\PY@tc##1{\textcolor[rgb]{0.25,0.50,0.50}{##1}}}
\expandafter\def\csname PY@tok@cm\endcsname{\let\PY@it=\textit\def\PY@tc##1{\textcolor[rgb]{0.25,0.50,0.50}{##1}}}
\expandafter\def\csname PY@tok@cpf\endcsname{\let\PY@it=\textit\def\PY@tc##1{\textcolor[rgb]{0.25,0.50,0.50}{##1}}}
\expandafter\def\csname PY@tok@c1\endcsname{\let\PY@it=\textit\def\PY@tc##1{\textcolor[rgb]{0.25,0.50,0.50}{##1}}}
\expandafter\def\csname PY@tok@cs\endcsname{\let\PY@it=\textit\def\PY@tc##1{\textcolor[rgb]{0.25,0.50,0.50}{##1}}}

\def\PYZbs{\char`\\}
\def\PYZus{\char`\_}
\def\PYZob{\char`\{}
\def\PYZcb{\char`\}}
\def\PYZca{\char`\^}
\def\PYZam{\char`\&}
\def\PYZlt{\char`\<}
\def\PYZgt{\char`\>}
\def\PYZsh{\char`\#}
\def\PYZpc{\char`\%}
\def\PYZdl{\char`\$}
\def\PYZhy{\char`\-}
\def\PYZsq{\char`\'}
\def\PYZdq{\char`\"}
\def\PYZti{\char`\~}
% for compatibility with earlier versions
\def\PYZat{@}
\def\PYZlb{[}
\def\PYZrb{]}
\makeatother


    % Exact colors from NB
    \definecolor{incolor}{rgb}{0.0, 0.0, 0.5}
    \definecolor{outcolor}{rgb}{0.545, 0.0, 0.0}



    
    % Prevent overflowing lines due to hard-to-break entities
    \sloppy 
    % Setup hyperref package
    \hypersetup{
      breaklinks=true,  % so long urls are correctly broken across lines
      colorlinks=true,
      urlcolor=urlcolor,
      linkcolor=linkcolor,
      citecolor=citecolor,
      }
    % Slightly bigger margins than the latex defaults
    
    \geometry{verbose,tmargin=1in,bmargin=1in,lmargin=1in,rmargin=1in}
    
    

    \begin{document}
    
    
    \maketitle
    
    

    
    \begin{Verbatim}[commandchars=\\\{\}]
{\color{incolor}In [{\color{incolor}1}]:} \PY{k+kn}{import} \PY{n+nn}{platform}
        \PY{k}{if} \PY{n}{platform}\PY{o}{.}\PY{n}{system}\PY{p}{(}\PY{p}{)} \PY{o}{==} \PY{l+s+s1}{\PYZsq{}}\PY{l+s+s1}{Linux}\PY{l+s+s1}{\PYZsq{}}\PY{p}{:}
            \PY{o}{\PYZpc{}}\PY{k}{run} \PYZsq{}/home/jonasmaziero/Dropbox/GitHub/algebra\PYZus{}linear/00init.ipynb\PYZsq{}
        \PY{k}{else}\PY{p}{:}
            \PY{o}{\PYZpc{}}\PY{k}{run} \PYZsq{}/Users/jonasmaziero/Dropbox/GitHub/algebra\PYZus{}linear/00init.ipynb\PYZsq{}
\end{Verbatim}


    \section{Álgebra Linear}\label{uxe1lgebra-linear}

É o estudo de espaços vetoriais e de operações lineares nesses espaços.
Já em sua versão mais básica, a AL tem aplicações diversas em ciência.
Uma boa descrição sobre o motivo de definirmos um espaço vetorial, ou
grupo, como o faremos, pode ser visto no playlist de álgebra abstrata da
Socratica: https://youtu.be/IP7nW\_hKB7I.

    \subsection{Espaço vetorial}\label{espauxe7o-vetorial}

Um conjunto de objetos \(\{|v\rangle\}\) forma um espaço vetorial \(V\)
se existirem a operação de \emph{soma de vetores}
\(+:V\text{x}V\rightarrow V\), que leva dois vetores em um vetor de
\(V\), com as seguintes propriedades: * Comutatividade:
\(|v\rangle+|w\rangle=|w\rangle+|v\rangle\text{, }\forall|v\rangle,|w\rangle\in V\),
* Associatividade:
\(|v\rangle+(|w\rangle+|x\rangle)=(|v\rangle+|w\rangle)+|x\rangle\text{, }\forall|v\rangle,|w\rangle,|x\rangle\in V\),
* Existe o elemento nulo \(|\oslash\rangle\) tal que:
\(|v\rangle+|\oslash\rangle=|v\rangle\text{, }\forall|v\rangle\in V\), *
Existe o elemento inverso \(|v^{-1}\rangle\) tal que:
\(|v\rangle+|v^{-1}\rangle=|\oslash\rangle\text{, }\forall|v\rangle\in V\),
e a operação de \emph{multiplicação por escalar}
\(*:\mathbb{F}\text{x}V\rightarrow V\), herdada da multiplicação em
\(\mathbb{F}\) e que leva um escalar do campo escalar associado e um
vetor de \(V\) em um vetor de \(V\), com as seguintes propriedades: * Se
\(\mathbb{1}_{F}\) é a identidade para a multiplicação em
\(\mathbb{F}\), então
\(\mathbb{1}_{F}*|v\rangle=|v\rangle\text{, }\forall|v\rangle\in V\), *
Associatividade para produto dos escalares:
\((a*b)*|v\rangle=a*(b*|v\rangle)\text{, }\forall a,b\in\mathbb{F}\text{ e }\forall|v\rangle\in V\),
* Distributividade para a soma dos vetores:
\(a*(|v\rangle+|w\rangle)=a*|v\rangle+a*|w\rangle\text{, }\forall a\in\mathbb{F}\text{ e }\forall|v\rangle,|w\rangle\in V\).
* Distributividade para a soma de escalares:
\((a+b)*|v\rangle = a*|v\rangle+b*|v\rangle\text{ }\forall a,b\in\mathbb{F}\text{ e }|v\rangle\in V\).

    \subsubsection{Proposição}\label{proposiuxe7uxe3o}

Seja \(\mathbb{0}_{\mathbb{F}}+a=a\text{ }\forall a\in \mathbb{F}\).
Então
\(\mathbb{0}_{\mathbb{F}}*|v\rangle=|\oslash\rangle\text{ }\forall|v\rangle\in V\).
\#\#\#\# Verificação

\begin{align}
\mathbb{0}_{\mathbb{F}}*|v\rangle & = \mathbb{0}_{\mathbb{F}}*|v\rangle+|\oslash\rangle = \mathbb{0}_{\mathbb{F}}*|v\rangle+|v\rangle+|v^{-1}\rangle \\
& = \mathbb{0}_{\mathbb{F}}*|v\rangle+\mathbb{1}_{\mathbb{F}}*|v\rangle+|v^{-1}\rangle 
= (\mathbb{0}_{\mathbb{F}}+\mathbb{1}_{\mathbb{F}})*|v\rangle+|v^{-1}\rangle \\ 
& = \mathbb{1}_{\mathbb{F}}*|v\rangle+|v^{-1}\rangle = |v\rangle+|v^{-1}\rangle \\
& = |\oslash\rangle.
\end{align}

    \paragraph{Exemplos}\label{exemplos}

Para o conjunto de listas com \(n\) números complexos, denotado por
\(\mathbb{C}^{n}\), as operações de soma de vetores e de multiplicação
por escalar são definidas por:

\begin{align}
& |v\rangle+|w\rangle:=\begin{bmatrix} v_{1} \\ v_{2} \\ \vdots \\ v_{n} \end{bmatrix} + \begin{bmatrix} w_{1} \\ w_{2} \\ \vdots \\ w_{n} \end{bmatrix} := \begin{bmatrix} v_{1}+w_{1} \\ v_{2}+w_{2} \\ \vdots \\ v_{n}+w_{n} \end{bmatrix} \\
& a*|v\rangle = a*\begin{bmatrix} v_{1} \\ v_{2} \\ \vdots \\ v_{n} \end{bmatrix} := \begin{bmatrix} a*v_{1} \\ a*v_{2} \\ \vdots \\ a*v_{n} \end{bmatrix},
\end{align}

com \(a,v_{j},w_{k}\in\mathbb{C}\) e \(+\) e \(*\) nas matrizes são as
usuais operações de soma e multiplicação de números complexos. Aplicando
as propriedades dessas operações, pode-se verificar que essas definições
satisfazem as propriedades acima e tornam assim \(\mathbb{C}^{n}\) em um
espaço vetorial. \emph{OBS:} \(\mathbb{R}^{n}\) é um caso particular de
\(\mathbb{C}^{n}\), e é também um espaço vetorial.

\textbf{Exercício:} Forneça as conhecidas definições de soma e
multiplicação por escalar que fazem de \(\mathbb{C}^{n\text{x}n}\) um
espaço vetorial.

    \subsection{Combinação e independência
linear}\label{combinauxe7uxe3o-e-independuxeancia-linear}

Dizemos que um vetor \(|v\rangle\in V\), com \(V\) sendo um espaço
vetorial, é uma \emph{combinação linear} de um conjunto de vetores
\(|v_{j}\rangle\in V\) se existem escalares do campo escalar associado
\(a_{j}\in\mathbb{F}\) tais que

\begin{equation}
|v\rangle = \sum_{j}a_{j}|v_{j}\rangle.
\end{equation}

Um conjunto de \(n\) vetores \(|w_{j}\rangle\in V\) é um conjunto
\emph{linearmente independente} (LI) se nenhum desses vetores pode ser
escrito como uma combinação linear dos outros. Colocado de outra forma,
\(\{|w_{j}\rangle\}_{j=1}^{n}\) é LI se a única maneira de satisfazer a
igualdade

\begin{equation}
\sum_{j}a_{j}|w_{j}\rangle=|\oslash\rangle,
\end{equation}

para \(a_{j}\in\mathbb{F}\), é com todos os coeficientes da combinação
linear nulos, i.e., \(a_{j}=0\) para \(j=1,\cdots,n\).

\emph{OBS:} Se um conjunto de vetores não é LI, então este é dito
linearmente dependente (LD). Note que nesse caso existe pelo menos um
coeficiente \(a_{k}\) não nulo, e podemos escrever
\(|w_{k}\rangle=\sum_{j\ne k}(-a_{j}/a_{k})|w_{j}\rangle\).

\paragraph{Exemplos}\label{exemplos}

Para \(\mathbb{C}^{2}\) temos que
\(|w_{1}\rangle=\begin{bmatrix} i & 1 \end{bmatrix}^{T}\) e
\(|w_{2}\rangle=\begin{bmatrix} 3i & 1 \end{bmatrix}^{T}\) são LI. Já
\(|w_{1}\rangle\) e
\(|w_{3}\rangle=\begin{bmatrix} 3i & 3 \end{bmatrix}^{T}\) são LD pois
\(-3|w_{1}\rangle+|w_{3}\rangle=|\oslash\rangle\).

    \subsection{Critério para determinar se um conjunto de vetores é
LI}\label{crituxe9rio-para-determinar-se-um-conjunto-de-vetores-uxe9-li}

Lembrando, se só obtemos
\(\sum_{j=1}^{n}a_{j}|w_{j}\rangle=|\oslash\rangle\) se
\(|a\rangle=\begin{bmatrix} a_{1} & \cdots & a_{n} \end{bmatrix}^{T}=\begin{bmatrix} 0 & \cdots & 0 \end{bmatrix}^{T}=|\oslash\rangle\),
então o conjunto de vetores \(\{|w_{j}\rangle\}_{j=1}^{n}\) é LI, senão
é LD. Vamos transformar essa equação em uma equação matricial definindo
\(|w_{j}\rangle=\begin{bmatrix} w_{1,j} & \cdots & w_{n,j} \end{bmatrix}^{T}\).
Assim

\begin{align}
|\oslash\rangle & = \sum_{j=1}^{n}a_{j}|w_{j}\rangle = \sum_{j=1}^{n}a_{j}\begin{bmatrix} w_{1j} \\ \vdots \\ w_{nj} \end{bmatrix} = \sum_{j=1}^{n}\begin{bmatrix} a_{j}w_{1j} \\ \vdots \\ a_{j}w_{nj} \end{bmatrix} = \begin{bmatrix} \sum_{j=1}^{n}w_{1j}a_{j} \\ \vdots \\ \sum_{j=1}^{n}w_{nj}a_{j} \end{bmatrix} \\
& = \begin{bmatrix} w_{1,1} & \cdots & w_{1,n} \\ \vdots & \vdots & \vdots \\  w_{n,1} & \cdots & w_{n,n} \end{bmatrix}\begin{bmatrix} a_{1} \\ \vdots \\ a_{n} \end{bmatrix} =: W|a\rangle.
\end{align}

Vemos assim que se a matriz

\begin{equation}
W=\begin{bmatrix} |w_{1}\rangle & \cdots & |w_{n}\rangle \end{bmatrix}
\end{equation}

possuir inversa, i.e., se \(\det(W)\ne 0\), então

\begin{equation}
W|a\rangle=|\oslash\rangle\Rightarrow W^{-1}W|a\rangle=\mathbb{I}_{n}|a\rangle=|a\rangle=W^{-1}|\oslash\rangle=|\oslash\rangle,
\end{equation}

o que implica que o conjunto de vetores é LI. Se tivermos \(\det(W)=0\)
o conjunto de vetores é LD. \emph{OBS:} Esse critério se torna óbvio se
lembrarmos que o determinante de uma matriz é nulo se uma (ou mais)
coluna(s) dessa matriz é uma combinação linear de outras das suas
colunas. Essa fato equivale, no presente contexto, a um (ou mais) dos
vetores \(|w_{j}\rangle\) ser uma combinação linear de outros desses
vetores.

    \paragraph{Exemplos}\label{exemplos}

Considere o conjunto de vetores
\(\{|w_{1}\rangle=\begin{bmatrix} 2 & 3 \end{bmatrix}^{T},|w_{2}\rangle=\begin{bmatrix} 5 & 7 \end{bmatrix}^{T}\}\).
Teremos

\begin{equation}
\det(W)=\det\begin{bmatrix} |w_{1}\rangle & |w_{2}\rangle \end{bmatrix}=\det\begin{bmatrix} 2 & 5 \\ 3 & 7 \end{bmatrix} = -1 \ne 0.
\end{equation}

Portanto \(\{|w_{1}\rangle,|w_{2}\rangle\}\) é LI.

Consideremos agora outro conjunto de vetores
\(\left\{|w_{1}\rangle=\begin{bmatrix} 2 & 3 \end{bmatrix}^{T},|w_{2}\rangle=\begin{bmatrix} 5 & 7 \end{bmatrix}^{T},|w_{3}\rangle=\begin{bmatrix} 11 & 13 \end{bmatrix}^{T}\right\}\).
Teremos

\begin{equation}
\det(W)=\det\begin{bmatrix} |w_{1}\rangle & |w_{2}\rangle & |w_{3}\rangle\end{bmatrix}=\det\begin{bmatrix} 2 & 5 & 11 \\ 3 & 7 & 13 \end{bmatrix}.
\end{equation}

Como olhamos somente para o determinante de matrizes quadradas, vamos
ver esses vetores como sendo vetores de \(\mathbb{R}^{3}\) com
componentes não nulas somente em \(\mathbb{R}^{2}\), i.e.,
\(\left\{|w'_{1}\rangle=\begin{bmatrix} 2 & 3 & 0 \end{bmatrix}^{T},|w'_{2}\rangle=\begin{bmatrix} 5 & 7 & 0 \end{bmatrix}^{T},|w'_{3}\rangle=\begin{bmatrix} 11 & 13 & 0 \end{bmatrix}^{T}\right\}\).
Nesse caso teríamos

\begin{equation}
\det(W')=\det\begin{bmatrix} |w'_{1}\rangle & |w'_{2}\rangle & |w'_{3}\rangle\end{bmatrix}=\det\begin{bmatrix} 2 & 5 & 11 \\ 3 & 7 & 13 \\ 0 & 0 & 0 \end{bmatrix}=0,
\end{equation}

pela expansão em cofatores na última linha. Portanto
\(\{|w'_{1}\rangle,|w'_{2}\rangle,|w'_{2}\rangle\}\), e
\(\{|w_{1}\rangle,|w_{2}\rangle,|w_{2}\rangle\}\), é LD. Esse resultado
pode ser verificado usando
\(|w_{3}\rangle=\alpha|w_{1}\rangle+\beta|w_{2}\rangle\) e obtendo os
coeficientes. Teremos

\begin{align}
& \begin{bmatrix} 2 & 5 \\ 3 & 7 \end{bmatrix}\begin{bmatrix} \alpha \\ \beta \end{bmatrix} = \begin{bmatrix} 11 \\ 13 \end{bmatrix} \Rightarrow \begin{bmatrix} \alpha \\ \beta \end{bmatrix} = \begin{bmatrix} 2 & 5 \\ 3 & 7 \end{bmatrix}^{-1}\begin{bmatrix} 11 \\ 13 \end{bmatrix} \\
& \Rightarrow \begin{bmatrix} \alpha \\ \beta \end{bmatrix} = \begin{bmatrix} -7 & 5 \\ 3 & -2 \end{bmatrix}\begin{bmatrix} 11 \\ 13 \end{bmatrix}=\begin{bmatrix} -12 \\ 7 \end{bmatrix}.
\end{align}

    \begin{Verbatim}[commandchars=\\\{\}]
{\color{incolor}In [{\color{incolor}2}]:} \PY{n}{w1} \PY{o}{=} \PY{n}{Matrix}\PY{p}{(}\PY{p}{[}\PY{p}{[}\PY{l+m+mi}{2}\PY{p}{]}\PY{p}{,}\PY{p}{[}\PY{l+m+mi}{3}\PY{p}{]}\PY{p}{]}\PY{p}{)}
        \PY{n}{w2} \PY{o}{=} \PY{n}{Matrix}\PY{p}{(}\PY{p}{[}\PY{p}{[}\PY{l+m+mi}{5}\PY{p}{]}\PY{p}{,}\PY{p}{[}\PY{l+m+mi}{7}\PY{p}{]}\PY{p}{]}\PY{p}{)}
        \PY{n}{w3} \PY{o}{=} \PY{n}{Matrix}\PY{p}{(}\PY{p}{[}\PY{p}{[}\PY{l+m+mi}{11}\PY{p}{]}\PY{p}{,}\PY{p}{[}\PY{l+m+mi}{13}\PY{p}{]}\PY{p}{]}\PY{p}{)}
        \PY{n}{A} \PY{o}{=} \PY{n}{Matrix}\PY{p}{(}\PY{p}{[}\PY{p}{[}\PY{l+m+mi}{2}\PY{p}{,}\PY{l+m+mi}{5}\PY{p}{]}\PY{p}{,}\PY{p}{[}\PY{l+m+mi}{3}\PY{p}{,}\PY{l+m+mi}{7}\PY{p}{]}\PY{p}{]}\PY{p}{)}
        \PY{c+c1}{\PYZsh{}A.inv()*w3}
        \PY{o}{\PYZhy{}}\PY{l+m+mi}{12}\PY{o}{*}\PY{n}{w1}\PY{o}{+}\PY{l+m+mi}{7}\PY{o}{*}\PY{n}{w2}  \PY{c+c1}{\PYZsh{} verificação}
\end{Verbatim}

\texttt{\color{outcolor}Out[{\color{outcolor}2}]:}
    
    $$\left[\begin{matrix}11\\13\end{matrix}\right]$$

    

    \textbf{Exercício:} Verifique que
\(\{|w_{1}\rangle=\begin{bmatrix} 17 & 19 \end{bmatrix}^{T},|w_{2}\rangle=\begin{bmatrix} 23 & 29 \end{bmatrix}^{T}\}\)
é LI e que
\(\{|w_{1}\rangle=\begin{bmatrix} 17 & 19 \end{bmatrix}^{T},|w_{2}\rangle=\begin{bmatrix} 23 & 29 \end{bmatrix}^{T},|w_{3}\rangle=\begin{bmatrix} 31 & 37 \end{bmatrix}^{T}\}\)
é LD.

    \subsubsection{Extensão}\label{extensuxe3o}

A extensão de um conjunto de vetores
\(\{|v_{j}\rangle\}_{j=1}^{n}\subseteq V\) são todos os vetores que são
obtidos através de combinações lineares desse conjunto de vetores, i.e.,

\begin{equation}
ext(\{|v_{j}\rangle\}_{j=1}^{n}) = \left\{\sum_{j=1}^{n}a_{j}|v_{j}\rangle \text{ para } a_{j}\in\mathbb{F}\right\}.
\end{equation}

\paragraph{Exemplo}\label{exemplo}

Considere \(|v_{1}\rangle=\begin{bmatrix} 1 & 0 \end{bmatrix}^{T}\) e
\(|v_{2}\rangle=\begin{bmatrix} 0 & 1 \end{bmatrix}^{T}\). Como
\(|v\rangle=\begin{bmatrix} a & b \end{bmatrix}^{T}=a|v_{1}\rangle+b|v_{2}\rangle\),
então \(ext(|v_{1}\rangle,|v_{2}\rangle)=\mathbb{C}^{2}\).

\textbf{Exercício:} Qual é a extensão dos vetores
\(|w_{1}\rangle=\begin{bmatrix} 1 & 1 \end{bmatrix}^{T}\) e
\(|w_{2}\rangle=\begin{bmatrix} 1 & -1 \end{bmatrix}^{T}\)?

    \subsection{Base}\label{base}

Se a extensão de um conjunto de vetores LI,
\(\{|v_{j}\rangle\}_{j=1}^{n}\subseteq V\), é todo o espaço vetorial,
i.e.,

\begin{equation}
ext(\{|v_{j}\rangle\}_{j=1}^{n}) = V,
\end{equation}

dizemos que esse conjunto de vetores forma uma base pra esse espaço
vetorial. \emph{OBS:} Note que nesse caso qualquer vetor de \(V\) pode
ser escrito como combinação linear de \(\{|v_{j}\rangle\}_{j=1}^{n}\).

\paragraph{Exemplo}\label{exemplo}

Os vetores \(|v_{1}\rangle\) e \(|v_{2}\rangle\) do tópico anterior
formam uma base para o espaço vetorial \(\mathbb{C}^{2}\).

\textbf{Exercício:} Os vetores \(|w_{1}\rangle\) e \(|w_{2}\rangle\) do
tópico anterior formam uma base para o espaço vetorial
\(\mathbb{C}^{2}\)?

    \subsection{Teorema}\label{teorema}

Seja \(\{|v_{j}\rangle\}_{j=1}^{r}\) um conjunto de vetores LI tal que
\(|v_{j}\rangle\in ext(\{|w_{k}\rangle\}_{k=1}^{s})\) para
\(j=1,\cdots,r\), com \(\{|w_{k}\rangle\}_{k=1}^{s}\) sendo também um
conjunto LI. Então \(r\le s\). \#\#\# Prova Pelo teorema, temos que cada
vetor \(|v_{j}\rangle\) pode ser escrito como uma combinação linear dos
vetores do conjunto \(\{|w_{k}\rangle\}_{k=1}^{s}\):

\begin{equation}
|v_{j}\rangle=\sum_{k=1}^{s}a_{j,k}|w_{k}\rangle,
\end{equation}

para \(j=1,\cdots,r\), com \(a_{j,k}\in\mathbb{F}\). Vamos assumir que
\(r>s\) e verificar que isso nos leva a uma \emph{contradição}.Todas as
esquações acima podem ser reescritas como uma única equação matricial:

\begin{equation}
\begin{bmatrix} |v_{1}\rangle \\ |v_{2}\rangle \\ \vdots \\ |v_{s}\rangle \\ |v_{s+1}\rangle \\ \vdots \\ |v_{r}\rangle \end{bmatrix}
= \begin{bmatrix} a_{1,1} & a_{1,2} & \cdots & a_{1,s} \\  a_{2,1} & a_{2,2} & \cdots & a_{2,s} \\ \vdots & \vdots & \vdots & \vdots \\ a_{s,1} & a_{s,2} & \cdots & a_{s,s} \\ a_{s,1} & a_{s+1,2} & \cdots & a_{s+1,s} \\  \vdots & \vdots & \vdots & \vdots  \\ a_{r,1} & a_{r,2} & \cdots & a_{r,s} \end{bmatrix}\begin{bmatrix} |w_{1}\rangle \\ |w_{2}\rangle \\ \vdots \\ |w_{s}\rangle \end{bmatrix}
\end{equation}

Aplicando eliminação Gaussiana, podemos colocar o bloco de cima da
matriz de coeficientes na forma "diagonal normalizada" (idetidade).
Lembrando, na eliminação Gaussiana trocamos uma linha por ela mais uma
constante multiplicada por outra linha. Teremos assim

\begin{equation}
\begin{bmatrix} |v'_{1}\rangle \\ |v'_{2}\rangle \\ \vdots \\ |v'_{s}\rangle \\ |v_{s+1}\rangle \\ \vdots \\ |v_{r}\rangle \end{bmatrix}
=\begin{bmatrix} 1 & 0 & \cdots & 0 \\  0 & 1 & \cdots & 0 \\  \vdots & \vdots & \vdots & \vdots \\ 0 & 0 & \cdots & 1 \\ a_{s,1} & a_{s+1,2} & \cdots & a_{s+1,s} \\  \vdots & \vdots & \vdots & \vdots  \\ a_{r,1} & a_{r,2} & \cdots & a_{r,s} \end{bmatrix}\begin{bmatrix} |w_{1}\rangle \\ |w_{2}\rangle \\ \vdots \\ |w_{s}\rangle \end{bmatrix},
\end{equation}

em que \(\{v'_{j}\}_{j=1}^{s}\) são combinações lineares dos vetores
\(\{v_{j}\}_{j=1}^{s}\). Pode-se ver que os vetores
\(\{|w_{j}\rangle\}_{j=1}^{s}\) não mudam pela eliminação Gaussiana pois
para \(j,l\in\{1,\cdots,s\}\) e \(c\in\mathbb{F}\),

\begin{align}
& |v_{j}\rangle\rightarrow|v_{j}\rangle+c|v_{l}\rangle \\
& \equiv \sum_{k=1}^{s}a_{j,k}|w_{k}\rangle +c  \sum_{k=1}^{s}a_{l,k}|w_{k}\rangle=\sum_{k=1}^{s}(a_{j,k}+ca_{l,k})|w_{k}\rangle. 
\end{align}

Depois de aplicado o procedimento da eliminação Gaussinana teremos

\begin{equation}
|v'_{j}\rangle=|w_{j}\rangle\text{ para } j=1,\cdots,s,
\end{equation}

com

\begin{equation}
|v'_{j}\rangle=\sum_{m=1}^{s}b_{j,m}|v_{m}\rangle\text{ para } b_{j,m}\in\mathbb{F}.
\end{equation}

Podemos escrever as linhas para \(n=s+1,\cdots,r\) como segue

\begin{align}
|v_{n}\rangle & = \sum_{k=1}^{s}a_{n,k}|w_{k}\rangle = \sum_{k=1}^{s}a_{n,k}|v'_{k}\rangle \\
& = \sum_{k=1}^{s}a_{n,k}\sum_{m=1}^{s}b_{k,m}|v_{m}\rangle = \sum_{m=1}^{s}\left(\sum_{k=1}^{s}a_{n,k}b_{k,m}\right)|v_{m}\rangle \\
& = \sum_{m=1}^{s}c_{n,m}|v_{m}\rangle 
\end{align}

com \(c_{n,m}\in\mathbb{F}\). Chegamos assim na conclusão contraditória
de que o conjunto \(\{|v_{j}\rangle\}_{j=1}^{r}\) é LD. Portanto nossa
suposição de que \(r>s\) deve estar errada e devemos ter que \(r\le s\).

    \paragraph{Corolário}\label{coroluxe1rio}

Se \(\{|v_{j}\rangle\}_{j=1}^{r}\) e \(\{|w_{k}\rangle\}_{k=1}^{s}\) são
duas bases para um espaço vetorial, então \(r=s\). \#\#\#\# Prova Como
os dois conjuntos são bases, estes conjuntos são, individualmente, LI e
cada vetor de um conjunto está na extensão do outro. Portanto, pelo
teorema anterior, devemos ter \(r\ge s\) e \(s\ge r\), que somente são
satisfeitas simultaneamente se \(r=s\).

    \subsubsection{Dimensão}\label{dimensuxe3o}

A dimensão de um espaço vetorial é definida como o número de seus
vetores LI que são necessários para gerar todos os seus vetores. Ou
seja, \(\dim(V)=n\) se uma base de \(V\) tiver \(n\) vetores.

    \subsection{Produto interno}\label{produto-interno}

Uma função
\(\langle \cdot|\cdot\rangle:V\text{x}V\rightarrow\mathbb{F}\) (leva
dois vetores em um escalar) é uma função produto interno se satisfaz as
seguintes propriedades: * \(\langle\cdot|\cdot\rangle\) é linear no
segundo argumento, i.e., para \(|v\rangle,|w_{j}\rangle\in V\) e
\(a_{j}\in\mathbb{F}\), devemos ter

\begin{equation}
\langle v|\left(\sum_{j}a_{j}|w_{j}\rangle\right) = \sum_{j}a_{j}\langle v|w_{j}\rangle.
\end{equation}

\begin{itemize}
\tightlist
\item
  Antisimetria por troca dos vetores, i.e., para
  \(|v\rangle,|w\rangle\in V\)

  \begin{equation}
  \langle v|w\rangle = \langle w|v\rangle^{*},
  \end{equation}

  onde \(*\) é o complexo conjugado.
\item
  Positividade, i.e., para \(|v\rangle\in V\)

  \begin{equation}
  \langle v|v\rangle\ge 0\text{ e }\langle v|v\rangle=0\Rightarrow |v\rangle=|\oslash\rangle.
  \end{equation}
\end{itemize}

\textbf{Exercício:} Mostre que o produto interno é anti-linear no
primeiro argumento, ou seja, se \(|w\rangle=\sum_{j}a_{j}|w_{j}\rangle\)
então

\begin{equation}
\langle w|v\rangle = \sum_{j}a_{j}^{*}\langle w_{j}|v\rangle.
\end{equation}

\textbf{Exercício:} Usando as propriedades do produto interno, verifique
que

\begin{equation}
|\langle v|w\rangle|^{2}=|\langle w|v\rangle|^{2}.
\end{equation}

    \paragraph{\texorpdfstring{Exemplo:
\(\mathbb{C}^{n}\)}{Exemplo: \textbackslash{}mathbb\{C\}\^{}\{n\}}}\label{exemplo-mathbbcn}

Para este espaço vetorial o produto interno é definido por

\begin{align}
\langle v|w\rangle & := |v\rangle^{\dagger}|w\rangle = \begin{bmatrix} v_{1}^{*} & v_{2}^{*} & \cdots & v_{n}^{*} \end{bmatrix}\begin{bmatrix} w_{1} \\ w_{2} \\ \vdots \\ w_{n} \end{bmatrix} \\
& = v_{1}^{*}w_{1}+v_{2}^{*}w_{2}+\cdots+v_{n}^{*}w_{n}= \sum_{j=1}^{n}v_{j}^{*}w_{j}.
\end{align}

    \begin{Verbatim}[commandchars=\\\{\}]
{\color{incolor}In [{\color{incolor}3}]:} \PY{k}{def} \PY{n+nf}{inner\PYZus{}product}\PY{p}{(}\PY{n}{d}\PY{p}{,}\PY{n}{v}\PY{p}{,}\PY{n}{w}\PY{p}{)}\PY{p}{:}
            \PY{n}{ip} \PY{o}{=} \PY{l+m+mi}{0}
            \PY{k}{for} \PY{n}{j} \PY{o+ow}{in} \PY{n+nb}{range}\PY{p}{(}\PY{l+m+mi}{0}\PY{p}{,}\PY{n}{d}\PY{p}{)}\PY{p}{:}
                \PY{n}{ip} \PY{o}{+}\PY{o}{=} \PY{n}{conjugate}\PY{p}{(}\PY{n}{v}\PY{p}{[}\PY{n}{j}\PY{p}{]}\PY{p}{)}\PY{o}{*}\PY{n}{w}\PY{p}{[}\PY{n}{j}\PY{p}{]}
            \PY{k}{return} \PY{n}{ip}
\end{Verbatim}


    \textbf{Exercício:} Calcule o produto interno entre
\(|v_{1}\rangle=\begin{bmatrix} 1 & i \end{bmatrix}^{T}\) e
\(|v_{2}\rangle=\begin{bmatrix} 1 & -i \end{bmatrix}^{T}\).

    Vamos \emph{verificar} que essa definição possui as propriedades acima.
Começamos considerando

\begin{equation}
|w\rangle = \sum_{j}a_{j}|w_{j}\rangle = \sum_{j}a_{j}\begin{bmatrix} |w_{j}\rangle_{1} \\ |w_{j}\rangle_{2} \\ \vdots \\ |w_{j}\rangle_{n} \end{bmatrix} = \begin{bmatrix} \sum_{j}a_{j}|w_{j}\rangle_{1} \\ \sum_{j}a_{j}|w_{j}\rangle_{2} \\ \vdots \\ \sum_{j}a_{j}|w_{j}\rangle_{n} \end{bmatrix}.
\end{equation}

Assim

\begin{align}
\langle v|w\rangle & = \sum_{k=1}^{n}v_{k}^{*}w_{k} = \sum_{k=1}^{n}v_{k}^{*}\sum_{j}a_{j}|w_{j}\rangle_{k} \\
& = \sum_{j}a_{j}\sum_{k=1}^{n}v_{k}^{*}|w_{j}\rangle_{k} = \sum_{j}a_{j}\langle v|w_{j}\rangle.
\end{align}

Agora

\begin{align}
\langle w|v\rangle^{*} & = \left(\sum_{j=1}^{n}w_{j}^{*}v_{j}\right)^{*} = \sum_{j=1}^{n}\left(w_{j}^{*}v_{j}\right)^{*} \\
& = \sum_{j=1}^{n}(w_{j}^{*})^{*}v_{j}^{*} = \sum_{j=1}^{n}v_{j}^{*}w_{j} \\
& = \langle v|w\rangle.
\end{align}

Finalmente

\begin{equation}
\langle v|v\rangle = \sum_{j=1}^{n}v_{j}^{*}v_{j} = \sum_{j=1}^{n}|v_{j}|^{2} \ge 0.
\end{equation}

Para que essa soma seja nula, devemos ter todos os \(v_{j}=0\). Isso
implicaria que \(|v\rangle=|\oslash\rangle\).

    \subsubsection{Traço}\label{trauxe7o}

O traço de uma matriz é uma função
\(Tr:\mathbb{C}^{n\text{x}n}\rightarrow\mathbb{C}\) definida como a soma
dos elementos na diagonal principal de uma matriz:

\begin{equation}
Tr(A):= \sum_{j=1}^{n}A_{j,j}.
\end{equation}

    \begin{Verbatim}[commandchars=\\\{\}]
{\color{incolor}In [{\color{incolor}4}]:} \PY{k}{def} \PY{n+nf}{trace}\PY{p}{(}\PY{n}{d}\PY{p}{,}\PY{n}{A}\PY{p}{)}\PY{p}{:}
            \PY{n}{tr} \PY{o}{=} \PY{l+m+mi}{0}
            \PY{k}{for} \PY{n}{j} \PY{o+ow}{in} \PY{n+nb}{range}\PY{p}{(}\PY{l+m+mi}{0}\PY{p}{,}\PY{n}{d}\PY{p}{)}\PY{p}{:}
                \PY{n}{tr} \PY{o}{+}\PY{o}{=} \PY{n}{A}\PY{p}{[}\PY{n}{j}\PY{p}{,}\PY{n}{j}\PY{p}{]}
            \PY{k}{return} \PY{n}{tr}
        \PY{c+c1}{\PYZsh{}trace(2,Matrix([[2,3],[5,7]]))}
\end{Verbatim}


    \subsubsection{Produto interno de
Hilbert-Schmidt}\label{produto-interno-de-hilbert-schmidt}

Essa é uma função
\(\langle\cdot|\cdot\rangle:\mathbb{C}^{m\text{x}n}\text{x}\mathbb{C}^{m\text{x}n}\rightarrow\mathbb{C}\)
definida como

\begin{align}
\langle A|B\rangle & = Tr(A^{\dagger}B) = \sum_{j=1}^{n}(A^{\dagger}B)_{j,j} \\
& = \sum_{j=1}^{n}\sum_{k=1}^{m}(A^{\dagger})_{j,k}B_{k,j} = \sum_{j=1}^{n}\sum_{k=1}^{m}A_{k,j}^{*}B_{k,j}.
\end{align}

    \begin{Verbatim}[commandchars=\\\{\}]
{\color{incolor}In [{\color{incolor}5}]:} \PY{k}{def} \PY{n+nf}{inner\PYZus{}product\PYZus{}hs}\PY{p}{(}\PY{n}{m}\PY{p}{,}\PY{n}{n}\PY{p}{,}\PY{n}{A}\PY{p}{,}\PY{n}{B}\PY{p}{)}\PY{p}{:} \PY{c+c1}{\PYZsh{} A=A(m,n), B=B(m,n)}
            \PY{n}{ip} \PY{o}{=} \PY{l+m+mi}{0}
            \PY{k}{for} \PY{n}{j} \PY{o+ow}{in} \PY{n+nb}{range}\PY{p}{(}\PY{l+m+mi}{0}\PY{p}{,}\PY{n}{n}\PY{p}{)}\PY{p}{:}
                \PY{k}{for} \PY{n}{k} \PY{o+ow}{in} \PY{n+nb}{range}\PY{p}{(}\PY{l+m+mi}{0}\PY{p}{,}\PY{n}{m}\PY{p}{)}\PY{p}{:}
                    \PY{n}{ip} \PY{o}{+}\PY{o}{=} \PY{n}{conjugate}\PY{p}{(}\PY{n}{A}\PY{p}{[}\PY{n}{k}\PY{p}{,}\PY{n}{j}\PY{p}{]}\PY{p}{)}\PY{o}{*}\PY{n}{B}\PY{p}{[}\PY{n}{k}\PY{p}{,}\PY{n}{j}\PY{p}{]}
            \PY{k}{return} \PY{n}{ip}
        \PY{c+c1}{\PYZsh{}inner\PYZus{}product\PYZus{}hs(2,Matrix([[2,3],[5,7]]),Matrix([[11,13],[19,23]]))}
\end{Verbatim}


    \textbf{Exercício:} Calcule o produto interno de Hilbert-Schmidt entre
as matrizes \(A=\begin{bmatrix} 0 & 1 \\ 1 & 0 \end{bmatrix}\) e
\(B=\begin{bmatrix} 0 & -i \\ i & 0 \end{bmatrix}\). \textbf{Exercício:}
Verificar que o produto interno de Hilbert-Schmidt satisfaz as
propriedades listadas acima, para uma função ser um produto interno.
\textbf{Exercício:} Verificar que para \(n=1\) o produto interno de
Hilbert-Schmidt é equivalente ao produto interno para
\(\mathbb{C}^{m}\).

    \subsubsection{Ortogonalidade}\label{ortogonalidade}

Dois vetores \(|v\rangle\) e \(|w\rangle\) são ditos ortogonais se

\begin{equation}
\langle v|w\rangle = 0.
\end{equation}

\subsubsection{Norma}\label{norma}

A norma ("tamanho") de um vetor \(|v\rangle\) é definida como a raiz
quadrada do produto interno do vetor com ele mesmo:

\begin{equation}
||v|| := \sqrt{\langle v|v\rangle}.
\end{equation}

    \begin{Verbatim}[commandchars=\\\{\}]
{\color{incolor}In [{\color{incolor}6}]:} \PY{k}{def} \PY{n+nf}{norm}\PY{p}{(}\PY{n}{d}\PY{p}{,}\PY{n}{v}\PY{p}{)}\PY{p}{:}
            \PY{k}{return} \PY{n}{sqrt}\PY{p}{(}\PY{n}{inner\PYZus{}product}\PY{p}{(}\PY{n}{d}\PY{p}{,}\PY{n}{v}\PY{p}{,}\PY{n}{v}\PY{p}{)}\PY{p}{)}
        \PY{c+c1}{\PYZsh{}float(norm(2,Matrix([[2],[3]])))}
\end{Verbatim}


    \begin{Verbatim}[commandchars=\\\{\}]
{\color{incolor}In [{\color{incolor}7}]:} \PY{k}{def} \PY{n+nf}{norm\PYZus{}hs}\PY{p}{(}\PY{n}{d}\PY{p}{,}\PY{n}{A}\PY{p}{)}\PY{p}{:}
            \PY{k}{return} \PY{n}{sqrt}\PY{p}{(}\PY{n}{inner\PYZus{}product\PYZus{}hs}\PY{p}{(}\PY{n}{d}\PY{p}{,}\PY{n}{A}\PY{p}{,}\PY{n}{A}\PY{p}{)}\PY{p}{)}
        \PY{c+c1}{\PYZsh{}float(norm\PYZus{}hs(2,Matrix([[2,3],[5,7]])))}
\end{Verbatim}


    \textbf{Exercício:} Calcule a norma do vetor
\(|v_{1}\rangle=\begin{bmatrix} 1 & i \end{bmatrix}^{T}\).
\textbf{Exercício:} Calcule a norma de Hilbert-Schmidt da matriz
\(B=\begin{bmatrix} 0 & -i \\ i & 0 \end{bmatrix}\).

    \subsection{Ortogonalização de
Gram-Schmidt}\label{ortogonalizauxe7uxe3o-de-gram-schmidt}

Dado um conjunto de vetores LI \(\{|v_{j}\rangle\}_{j=1}^{n}\), o
procedimento de GS descrito abaixo pode ser utilizado para obtermos um
conjunto \(\{|w_{j}\rangle\}_{j=1}^{n}\) ortonormal, i.e.,
\(\langle w_{j}|w_{k}\rangle=\delta_{j,k}\). O algoritmo é o seguinte:
1. Primeiro normalizamos \(|v_{1}\rangle\), i.e., fazemos

\begin{equation}
|w_{1}\rangle:=\frac{|v_{1}\rangle}{||v_{1}||}.
\end{equation}

Assim

\begin{equation}
||w_{1}||=\sqrt{\langle w_{1}|w_{1}\rangle} = \sqrt{\frac{\langle v_{1}|}{||v_{1}||}\frac{|v_{1}\rangle}{||v_{1}||}}=\sqrt{\frac{\langle v_{1}|v_{1}\rangle}{||v_{1}||^{2}}}=1.
\end{equation}

\begin{enumerate}
\def\labelenumi{\arabic{enumi}.}
\setcounter{enumi}{1}
\tightlist
\item
  Agora subtraímos a "componente" que \(|v_{2}\rangle\) possui na
  "direção" de \(|w_{1}\rangle\) e normalizamos o vetor obtido, i.e.,

  \begin{equation}
  |w_{2}\rangle := \frac{|v_{2}\rangle-\langle w_{1}|v_{2}\rangle|w_{1}\rangle}{||(|v_{2}\rangle-\langle w_{1}|v_{2}\rangle|w_{1}\rangle)||}.
  \end{equation}

  Esse vetor é normalizado e ortogonal a \(|w_{1}\rangle\):

  \begin{equation}
  \langle w_{1}|w_{2}\rangle \propto \langle w_{1}|v_{2}\rangle-\langle w_{1}|v_{2}\rangle\langle w_{1}|w_{1}\rangle=0.
  \end{equation}
\item
  Seguindo, subtraímos a "componente" que \(|v_{3}\rangle\) possui na
  "direção" de \(|w_{1}\rangle\) e de \(|w_{2}\rangle\) e normalizamos o
  vetor obtido, i.e.,

  \begin{equation}
  |w_{3}\rangle := \frac{|v_{3}\rangle-\langle w_{1}|v_{3}\rangle|w_{1}\rangle-\langle w_{2}|v_{3}\rangle|w_{2}\rangle}{||(|v_{3}\rangle-\langle w_{1}|v_{3}\rangle|w_{1}\rangle-\langle w_{2}|v_{3}\rangle|w_{2}\rangle)||}.
  \end{equation}

  Esse vetor é normalizado e ortogonal a \(|w_{1}\rangle\) e a
  \(|w_{2}\rangle\):

  \begin{align}
  & \langle w_{1}|w_{3}\rangle \propto \langle w_{1}|v_{3}\rangle-\langle w_{1}|v_{3}\rangle\langle w_{1}|w_{1}\rangle-\langle w_{2}|v_{3}\rangle\langle w_{1}|w_{2}\rangle=0 \\
  & \langle w_{2}|w_{3}\rangle \propto \langle w_{2}|v_{3}\rangle-\langle w_{1}|v_{3}\rangle\langle w_{2}|w_{1}\rangle-\langle w_{2}|v_{3}\rangle\langle w_{2}|w_{2}\rangle=0
  \end{align}
\item
  Para os outros \(j=4,\cdots,n\) vetores, usa a mesma ideia, i.e.,
  subtrai a componente na diração dos \(|w_{k<j}\rangle\) e normaliza:

  \begin{equation}
  |w_{j}\rangle := \frac{|v_{j}\rangle-\sum_{k=1}^{j-1}\langle w_{k}|v_{j}\rangle|w_{k}\rangle}{||(|v_{j}\rangle-\sum_{k=1}^{j-1}\langle w_{k}|v_{j}\rangle|w_{k}\rangle)||}.
  \end{equation}

  Abaixo está um programa que, fornecido o conjunto LI como as colunas
  de uma matrix \(A\), este retorna o conjunto ortonormal nas colunas da
  matriz \(B\).
\end{enumerate}

    \begin{Verbatim}[commandchars=\\\{\}]
{\color{incolor}In [{\color{incolor}1}]:} \PY{k}{def} \PY{n+nf}{gram\PYZus{}schmidt}\PY{p}{(}\PY{n}{m}\PY{p}{,}\PY{n}{n}\PY{p}{,}\PY{n}{A}\PY{p}{)}\PY{p}{:} \PY{c+c1}{\PYZsh{} A=A[m,n]}
            \PY{n}{B} \PY{o}{=} \PY{n}{zeros}\PY{p}{(}\PY{n}{m}\PY{p}{,}\PY{n}{n}\PY{p}{)}
            \PY{n}{B}\PY{p}{[}\PY{p}{:}\PY{p}{,}\PY{l+m+mi}{0}\PY{p}{]} \PY{o}{=} \PY{n}{A}\PY{p}{[}\PY{p}{:}\PY{p}{,}\PY{l+m+mi}{0}\PY{p}{]}\PY{o}{/}\PY{n}{norm}\PY{p}{(}\PY{n}{m}\PY{p}{,}\PY{n}{A}\PY{p}{[}\PY{p}{:}\PY{p}{,}\PY{l+m+mi}{0}\PY{p}{]}\PY{p}{)}
            \PY{k}{for} \PY{n}{j} \PY{o+ow}{in} \PY{n+nb}{range}\PY{p}{(}\PY{l+m+mi}{1}\PY{p}{,}\PY{n}{n}\PY{p}{)}\PY{p}{:}
                \PY{n}{B}\PY{p}{[}\PY{p}{:}\PY{p}{,}\PY{n}{j}\PY{p}{]} \PY{o}{=} \PY{n}{A}\PY{p}{[}\PY{p}{:}\PY{p}{,}\PY{n}{j}\PY{p}{]}
                \PY{k}{for} \PY{n}{k} \PY{o+ow}{in} \PY{n+nb}{range}\PY{p}{(}\PY{l+m+mi}{0}\PY{p}{,}\PY{n}{j}\PY{p}{)}\PY{p}{:}
                    \PY{n}{B}\PY{p}{[}\PY{p}{:}\PY{p}{,}\PY{n}{j}\PY{p}{]} \PY{o}{\PYZhy{}}\PY{o}{=} \PY{n}{inner\PYZus{}product}\PY{p}{(}\PY{n}{m}\PY{p}{,}\PY{n}{B}\PY{p}{[}\PY{p}{:}\PY{p}{,}\PY{n}{k}\PY{p}{]}\PY{p}{,}\PY{n}{A}\PY{p}{[}\PY{p}{:}\PY{p}{,}\PY{n}{j}\PY{p}{]}\PY{p}{)}\PY{o}{*}\PY{n}{B}\PY{p}{[}\PY{p}{:}\PY{p}{,}\PY{n}{k}\PY{p}{]}
                \PY{n}{B}\PY{p}{[}\PY{p}{:}\PY{p}{,}\PY{n}{j}\PY{p}{]} \PY{o}{/}\PY{o}{=} \PY{n}{norm}\PY{p}{(}\PY{n}{m}\PY{p}{,}\PY{n}{B}\PY{p}{[}\PY{p}{:}\PY{p}{,}\PY{n}{j}\PY{p}{]}\PY{p}{)}
            \PY{k}{return} \PY{n}{B}
        \PY{c+c1}{\PYZsh{}gram\PYZus{}schmidt(2,2,Matrix([[1,1],[1,\PYZhy{}1]]))}
\end{Verbatim}


    \textbf{Exercício:} Use o procedimento de Gram-Schmidt para
ortonormalizar
\(|v_{1}\rangle = \begin{bmatrix} 1 & 1\end{bmatrix}^{T}\) e
\(|v_{2}\rangle = \begin{bmatrix} 1 & -1\end{bmatrix}^{T}\).

    \paragraph{Decomposição de um vetor em uma base
ortonormal}\label{decomposiuxe7uxe3o-de-um-vetor-em-uma-base-ortonormal}

Consideremos um vetor qualquer \(|v\rangle\in V\) decomposto em uma base
ortonormal \(\{|b_{j}\rangle\}_{j=1}^{\dim V}\) para \(V\) como

\begin{equation}
|v\rangle = \sum_{j=1}^{\dim V}v_{j}|b_{j}\rangle,
\end{equation}

com \(v_{j}\in\mathbb{F}\). Teremos então que

\begin{align}
\langle b_{k}|v\rangle & = \langle b_{k}|(\sum_{j=1}^{\dim V}v_{j}|b_{j}\rangle) = \sum_{j=1}^{\dim V}v_{j}\langle b_{k}|b_{j}\rangle \\
& = \sum_{j=1}^{\dim V}v_{j}\delta_{k,j} = v_{k}.
\end{align}

Ou seja,

\begin{equation}
|v\rangle = \sum_{j=1}^{\dim V}\langle b_{j}|v\rangle|b_{j}\rangle.
\end{equation}

    \subsection{Espaço de Hilbert}\label{espauxe7o-de-hilbert}

Consideraremos estes espaços como sendo espaços vetoriais munidos de uma
função produto interno. O exemplo de espaço de Hilbert mais simples em
Mecânica (MQ) é aquele usado para descrever "sistemas discretos":

\begin{equation}
\mathcal{H}=\left(\mathbb{c}^{n},\langle\psi|\phi\rangle=|\psi\rangle^{\dagger}|\phi\rangle\right).
\end{equation}

Já para tópicos avançados da MQ, tais como computação quântica, é comum
usarmos

\begin{equation}
\mathcal{H}=\left(\mathbb{c}^{n\text{x}n},\langle A|B\rangle=Tr(A^{\dagger}B)\right).
\end{equation}

Em ambos os casos aparece frequentemente o uso do espaço de funções

\begin{equation}
\mathcal{H}=\left(f,g:\mathbb{c}^{n}\rightarrow\mathbb{c},\langle f(\vec{x})|g(\vec{x})\rangle=\int d\vec{x}f^{\dagger}(\vec{x})g(\vec{x})\right).
\end{equation}

    \subsubsection{Desigualdade de
Cauchy-Schwarz}\label{desigualdade-de-cauchy-schwarz}

\paragraph{Teorema}\label{teorema}

Para quaisquer dois vetores \(|v\rangle,|w\rangle\in\mathcal{H}\), segue
que

\begin{equation}
\langle v|v\rangle\langle w|w\rangle \ge \langle v|w\rangle\langle w|v\rangle.
\end{equation}

\paragraph{Prova}\label{prova}

Para provar essa desigualdade, utilizaremos a positividade do produto
interno, i.e.,
\(\langle x|x\rangle\ge 0\text{ }\forall|x\rangle\in\mathcal{H}\), com a
definição apropriada para este vetor:
\(|x\rangle=|v\rangle+c|w\rangle\), com \(c\in\mathbb{F}\). Teremos
assim que

\begin{align}
0 & \le \langle x|x\rangle \\
& = (|v\rangle+c|w\rangle,|v\rangle+c|w\rangle) \\
& = \langle v|v\rangle + c^{*}\langle w|v\rangle + c\langle v|w\rangle + cc^{*}\langle w|w\rangle. 
\end{align}

Tendo em vista o que queremos provar, definiremos
\(c:=-\langle w|v\rangle/\langle w|w\rangle\). Vem assim que

\begin{align}
0 & \le \langle v|v\rangle - \frac{\langle w|v\rangle^{*}}{\langle w|w\rangle}\langle w|v\rangle - \frac{\langle w|v\rangle}{\langle w|w\rangle}\langle v|w\rangle + \frac{\langle w|v\rangle}{\langle w|w\rangle}\frac{\langle w|v\rangle^{*}}{\langle w|w\rangle}\langle w|w\rangle \\
& = \langle v|v\rangle - \frac{|\langle w|v\rangle|^{2}}{\langle w|w\rangle} - \frac{|\langle w|v\rangle|^{2}}{\langle w|w\rangle} + \frac{|\langle w|v\rangle|^{2}}{\langle w|w\rangle^{2}}\langle w|w\rangle \\
& = \langle v|v\rangle - \frac{|\langle w|v\rangle|^{2}}{\langle w|w\rangle}.
\end{align}

Multiplicando toda equação por \(\langle w|w\rangle\) completaremos a
verificação:

\begin{equation}
\langle v|v\rangle\langle w|w\rangle - |\langle v|w\rangle|^{2} \ge 0.
\end{equation}

\textbf{Exercício:} Mostre que a igualdade é obtida na desigualdade de
Cauchy-Schwarz se e somente se os dois vetores forem proporcionais
(i.e., se tiverem a mesma direção).

    \subsection{Subespaço vetorial}\label{subespauxe7o-vetorial}

Se \(V\) é um espaço vetorial, então \(W\subseteq V\) é um sub-espaço de
\(V\) se também for um espaço vetorial sob as operações de soma e
multiplicação por escalar de \(V\).

\emph{Exemplo:} \(\mathbb{C}^{2}\) é um subespaço de \(\mathbb{C}^{3}\).

\textbf{Exercício:} Forneça um exemplo de subespaço para o espaço
vetorial \(\mathbb{C}^{4\text{x}4}\). Qual é a dimensão desse espaço
vetorial? Qual é dimensão do subespaço que você escolheu como exemplo?


    % Add a bibliography block to the postdoc
    
    
    
    \end{document}
